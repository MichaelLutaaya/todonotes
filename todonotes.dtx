% \iffalse meta-comment
%
% Copyright (C) 2008 by Henrik Skov Midtiby <henrikmidtiby@gmail.com>
% -------------------------------------------------------
% 
% This file may be distributed and/or modified under the
% conditions of the LaTeX Project Public License, either version 1.2
% of this license or (at your option) any later version.
% The latest version of this license is in:
%
%    http://www.latex-project.org/lppl.txt
%
% and version 1.2 or later is part of all distributions of LaTeX 
% version 1999/12/01 or later.
%
% \fi
%
% \iffalse
%<*driver>
\ProvidesFile{todonotes.dtx}
%</driver>
%<package>\NeedsTeXFormat{LaTeX2e}[1999/12/01]
%<package>\ProvidesPackage{todonotes}
%<*package>
    [2009/01/09 .dtx Todonotes source and documentation.]
%</package>
%
%<*driver>
\documentclass{ltxdoc}
\usepackage{wrapfig}
\usepackage[colorlinks, linkcolor=black]{hyperref}
\usepackage[colorinlistoftodos, shadow]{todonotes}[2009/01/09]
\usepackage{amsmath}
\usepackage{setspace}
\setcounter{tocdepth}{2}
\EnableCrossrefs         
\CodelineIndex
\RecordChanges
\begin{document}
  \DocInput{todonotes.dtx}
%\iffalse
%  \PrintChanges
%  \PrintIndex
%\fi
\end{document}
%</driver>
% \fi
%
% \CheckSum{387}
%
% \CharacterTable
%  {Upper-case    \A\B\C\D\E\F\G\H\I\J\K\L\M\N\O\P\Q\R\S\T\U\V\W\X\Y\Z
%   Lower-case    \a\b\c\d\e\f\g\h\i\j\k\l\m\n\o\p\q\r\s\t\u\v\w\x\y\z
%   Digits        \0\1\2\3\4\5\6\7\8\9
%   Exclamation   \!     Double quote  \"     Hash (number) \#
%   Dollar        \$     Percent       \%     Ampersand     \&
%   Acute accent  \'     Left paren    \(     Right paren   \)
%   Asterisk      \*     Plus          \+     Comma         \,
%   Minus         \-     Point         \.     Solidus       \/
%   Colon         \:     Semicolon     \;     Less than     \<
%   Equals        \=     Greater than  \>     Question mark \?
%   Commercial at \@     Left bracket  \[     Backslash     \\
%   Right bracket \]     Circumflex    \^     Underscore    \_
%   Grave accent  \`     Left brace    \{     Vertical bar  \|
%   Right brace   \}     Tilde         \~}
%
%
% \changes{0.1}{2008/04/17}{The first version of the package}
% \changes{0.2}{2008/05/15}{Updated the option handling of the package}
% \changes{0.2.1}{2008/05/16}{Slightly modified by Kjell Magne Fauske to 
%             support notes in the left margin (for documentstyle book).}
% \changes{0.2.2}{2008/05/28}{Added a missingfigure command}
% \changes{0.2.3}{2008/06/10}{Made a dependency on the calc package}
% \changes{0.3}{2008/07/11}{Delayed the requirements for the hyperref package
%             untill begin document and added an optional argument
%             to the todo command for adding inline todonotes (Idea
%             from Patrick Toche)}
% \changes{0.3.1}{2008/07/15}{Added some options to the todo macro (Idea: Patrick
%             Toche) and made the \\listoftodos point at the inserted
%             todos and not only the current / previous section, 
%             subsection or figure using the \\phantomsection macro.}
% \changes{0.4}{2008/07/28}{Modified the behaviour of the inline todonotes, to
%             avoid empty lines around the inline todonotes.}
% \changes{0.4.1}{2008/07/28}{
%             Added the option colorinlistoftodos which inserts a
%             small box with the used fillcolor of the todonotes in
%             the list of todos.}
% \changes{0.4.2}{2008/07/30}{Fixed a bug with the disable option to the
%             package.}
% \changes{0.5}{2008/08/05}{Created a dtx file containing both source
%             code and documentation of the package}
% \changes{0.5.1}{2008/09/15}{Updated the documentation}
% \changes{0.5.2}{2008/10/20}{Fixed a bug that prevented the usage
%             of the option french for babel. Bug report by Thomas
%             Braun.}
% \changes{0.6}{2008/10/24}{Added the caption option to the todo
%             command.}
% \changes{0.6.1}{2008/11/07}{Added a new usecase with decreased line
%             spacing.}
% \changes{0.6.2}{2008/11/13}{Added a usecase by Fabrice Niessen.}
% \changes{0.7}{2008/11/22}{Added language options on request from
%             Peter Zimmermann.}
% \changes{0.7.1}{2008/12/07}{Reworked the color options for both
%             the whole package and the todo command. General code
%             clean up. Added the prependcaption package option.}
% \changes{0.7.2}{2008/12/27}{Avoid to change the fontsize inside
%             the list of todos, fixing a bug revealed by Vladimir
%             Zhuravlev.}
% \changes{0.7.3}{2008/12/30}{The localization options (danish and
%             german) and the disable options, were all flawed by
%             naming inconsistencies that made then break the
%             package. This have been fixed.}
% \changes{0.7.4}{2008/12/30}{Fixed a bug related to the caption
%             option for the todo command. Introduced a counter of
%             the number of inserted todos.}
% \changes{0.7.5}{2009/01/02}{Fixed a typo in a macroname.}
% \changes{0.7.6}{2009/01/04}{Added a textsize option for the
%             package and the prepend / noprepend option for the
%             todo command.}
% \changes{0.8}{2009/01/09}{Added three new translations french,
%             spanish and catalan thanks to Richard Dominique and
%             Joan Queralt. Improved the visual appearence of the
%             inserted notes (rounded corners and optional shadows) 
%             with code from Joan Queralt. Found an untranslated
%             textstring ''Figure'' in the source. Added a figwidth
%             option to the missingfigure command, patch by Paul
%             Ivanov.}
% \changes{0.8.1}{2009/01/12}{Added a space between the colored
%             square and the text in the list of todos. Added a new
%             usecase for marking old / new sections. Made the name
%             of listoftodos changeable.}
%
% \GetFileInfo{todonotes.dtx}
%
% \DoNotIndex{\newcommand,\newenvironment}
%
% \iffalse
% A macro for marking things todo before the next relase
% (typically update of documentation).
% \fi
% \newcommand{\donow}[1]{\todo[color=blue]{#1}}
% 
%
% \title{The \textsf{todonotes} package\thanks{This document
%   corresponds to \textsf{todonotes}~\fileversion, dated \filedate.}}
% \author{Henrik Skov Midtiby \\ \texttt{henrikmidtiby@gmail.com}}
%
% \maketitle
%
% \begin{abstract}
% The |todonotes| package allows you to insert to--do items in your
% document. At any point in the document a list of all the inserted
% to--do items can be listed with the |\listoftodos| command.
% \end{abstract}
%
% \tableofcontents
%
% \newpage
% \section{Introduction}
%
% The |todonotes| package makes three commands available to the
% user: |\todo[]{}|, |\missingfigure{}| and |\listoftodos|.
% |\todo[]{}| and |\missingfigure{}| makes it possible to insert
% notes in your document about things that has to be done later
% (todonotes \ldots).
% I developed the basic functionality of the package while I worked
% on my bachelor project.
% 
%
%
% \subsection{Usage}
%
% \DescribeMacro{\todo}
% My \index{\todo}most common usage of the todonotes package, is to 
% insert an uncustomized todonotes somewhere in a latex document. 
% An example of this usage is the command 
%
% |\todo{Make a cake \ldots}|, 
% 
% \noindent
% which renders like\todo{Make a cake \ldots}.
% The |\todo| command has this structure: 
% |\todo|\oarg{options}\marg{todo text}.
% The |todo text| is the text that will be shown in the todonote and
% in the list of todos. The optional argument |options|, allows the
% user to customize the appearance of the inserted todonotes.
% For a description of all the options see section
% \ref{subsecTodoOptions}.
%
%
% \DescribeMacro{\missingfigure}
% The |\missingfigure| command inserts an image containing an 
% attention sign and the given text.
% The command takes only one argument
% |\missingfigure|\marg{text}, a text string that could
% describe what the figure should consist of.
% An example of its usage could be 
%
% |\missingfigure{Make a sketch of the structure of a trebuchet.}| 
%
% \noindent
% which renders like.
%
% \missingfigure{Make a sketch of the structure of a trebuchet.}
%
%
% \DescribeMacro{\listoftodos}
% The |\listoftodos| command inserts a list of all the todos in the
% current document. |\listoftodos| takes no arguments.
% For this document the list of to--do's looks like.
% \listoftodos
%
%
% \subsection{Package options}
% \label{subsecPackageOptions}
% \DescribeMacro{disable}
% If the option |disable| is passed to the package, the macros
% usually defined by the package (|\todo|, |\listoftodos| and
% |\missingfigure|) are defined as macros with no effect, and thus
% all inserted notes are removed.
%
% \DescribeMacro{danish, german, french}
% \DescribeMacro{spanish, catalan}
% Use the danish or german translations of the text strings 
% ''List of todos'' and ''Missing figure''.
% The default is to use none of these options, which results in
% english text strings.
% 
% \DescribeMacro{colorinlistoftodos}
% Adds a small colored square in front of all items in the Todo
% list. The color of the square is the same as the fill color of the
% inserted todonote.
% This can be usefull if there are different types of todos 
% (insert reference, explain in detail, \ldots) where the color of
% the inserted todonote marks the type of todo.
%
% \DescribeMacro{color}
% \DescribeMacro{backgrouncolor}
% \DescribeMacro{linecolor}
% \DescribeMacro{bordercolor}
% These options sets the default colors for the todo command.
% There is three colors that can be specified. The border color
% (default |bordercolor=black|) around the inserted text, the color
% behind the inserted text (default |backgroundcolor=orange|) and
% the color of the line connecting the inserted textbox with the
% current location in the text (default |linecolor=orange|). 
% Setting the |color| option to |val| passes this value on to the 
% background and line color options.
% The specified colors must be valid according to the 
% |xcolor| package.
% 
% \DescribeMacro{textwidth}
% |textwidth=length| sets the width of a todo item in the margin to
% |length|. The width of inline todonotes will allways be the same 
% as the current line width.
%
% \DescribeMacro{textsize}
% |textsize=value| sets the default text size of the inserted
% todonotes to the given value.
% Value is the ''name'' of the used fontsize, eg. if the desired
% fontsize is |\tiny| use |textsize=tiny|. The default value is 
% |textsize=normalsize|.
%
% \DescribeMacro{prependcaption}
% The |prependcaption| option triggers a special behaviour of the
% |caption=val| option for the todo command, where the given value
% |val| is inserted in the inserted todonote.
% 
% \DescribeMacro{shadow}
% If the |shadow| option is given, the inserted todonotes will be
% displayed with a gray shadow.
% I expect that the option will trigger problems with tikz versions
% prior to 2.0.
%
% \DescribeMacro{dvistyle}
% When a document with todonotes is compiled with plain latex (to a
% dvi-file), there is an issue with the visual
% appearance\footnote{The problem is placement of text inside the
% colored boxes.}.
% The option |dvistyle| changes the appearance of the inserted
% todonotes to avoid this problem.
%
% 
%
% \subsection{Options for the todo command}
% \label{subsecTodoOptions}
% There are several options that can be given to the |\todo|
% command. All the options are described here and often I have
% included examples of the change in visual appearance.
%
% \DescribeMacro{color}
% \DescribeMacro{backgroundcolor}
% \DescribeMacro{linecolor}
% \DescribeMacro{bordercolor}
% These options set the color that is used in the current todo
% command.
% The color classes is the same as used in the color package
% options, see section \ref{subsecPackageOptions}.
% Default values can be set by the color
% options when the todonotes package is loaded.
% \todo[color=green!40]{And a green note}
% The todo notes inserted in this paragraph is created with the
% command
% |\todo[color=green!40]{And a green note}|.
% The color of the inserted note could be used to mark different
% types of tasks (insert references, explain something in detail,
% \ldots), this could be streamlined by defining new commands like
% below.
% \begin{verbatim}
% \newcommand{\insertref}[1]{\todo[color=green!40]{#1}}
% \newcommand{\explainindetail}[1]{\todo[color=red!40]{#1}}
% \end{verbatim}
% An example that uses all of the color options is given below
% \todo[linecolor=green!70!white, backgroundcolor=blue!20!white,
% bordercolor=red]{Anything but default colors}.
% \begin{verbatim}
% \todo[linecolor=green!70!white, backgroundcolor=blue!20!white,
% bordercolor=red]{Anything but default colors}.
% \end{verbatim}
%
%
% \DescribeMacro{line / noline}
% If you want to get rid of the line connecting the inserted note
% with the place in the text where the note occurs in the latex
% code, the option |noline| can be used.
% \todo[noline]{A note with no line connecting it to the placement
% in the original text.}
% |\todo[noline]{A note with no line ...}|
% \vspace{1.0cm}
% 
%
% \DescribeMacro{inline / noinline}
% It is possible to place a todonote inside the text instead of
% placing it in the margin, this could be desireable if the text in
% the note has a considerable length.
% |\todo[inline]{A todonote placed in the text}|
% \todo[inline]{A todonote placed in the text}

% \begin{wrapfigure}[1]{r}[20mm]{40mm}
% \begin{tikzpicture}
% \draw[red] (0, 0) circle(0.45);
% \draw[green] (1, 0) circle(0.45);
% \draw[blue] (2, 0) circle(0.45);
% \end{tikzpicture}
% \caption{A text explaining the image.
% \todo[inline]{Fill those circles \ldots}}
% \end{wrapfigure}
% Another usage for the inline option is when you want to add a
% todonote to a figure caption.
%
% \begin{verbatim}
% \begin{wrapfigure}{r}[20mm]{40mm}
% \begin{tikzpicture}
% \draw[red] (0, 0) circle(0.45);
% \draw[green] (1, 0) circle(0.45);
% \draw[blue] (2, 0) circle(0.45);
% \end{tikzpicture}
% \caption{A text explaining the image.
% \todo[inline]{Fill those circles \ldots}}
% \end{wrapfigure}
% \end{verbatim}
%
% \DescribeMacro{size}
% |size=val| changes the size of the text inside the todonote.
% The commands used to create the notes below are \\ \noindent
% |\todo[size=\Large]{A note with a large fontsize.}|
% and \\ \noindent
% |\todo[inline, size=\tiny]{Note with very small fontsize.}|.
% \todo[size=\Large]{A note with a large fontsize.}
% \todo[inline, size=\footnotesize]{Note with very small fontsize.}
%
% \DescribeMacro{list / nolist}
% When the option |nolist| is given, the todo item will not appear in
% the list of todos.
%
% \DescribeMacro{caption}
% The |caption| option enables the user to specify a short
% description of the todonote that are inserted in the list of
% todos instead of the full todonote text.
% \todo[caption={Short note}]{A very long and tedious note that
% cannot be on one line in the list of todos.}
% \begin{verbatim}
% \todo[caption={Short note}]{A very long and tedious note that
% cannot be on one line in the list of todos.}.
% \end{verbatim}
% The effect of this option is altered with the package option
% |prependcaption| or the |prepend| / |noprepend| option for the
% todo command.
%
% \DescribeMacro{prepend / noprepend}
% The options |prepend| and |noprepend| can be used for setting
% whether a given caption should be prepended to the todonote or
% not.
% Globally this can be set using the |prependcaption| option for the
% package.~\todo[prepend, caption={Short note with prepend}]{A very
% long and tedious note that cannot be on one line in the list of
% todos.} Below is the effect of the option shown using the code:
% \todo[noprepend, caption={Short note with noprepend}]{A very long
% and tedious note that cannot be on one line in the list of
% todos.}
% \begin{verbatim}
% \todo[prepend, caption={Short note with prepend}]{A very long and tedious 
% note that cannot be on one line in the list of todos.}.
% \todo[noprepend, caption={Short note with noprepend}]{A very long and 
% tedious note that cannot be on one line in the list of todos.}.
% \end{verbatim}
% \DescribeMacro{figwidth}
% The |figwidth=length| option sets the default width of the figure
% inserted by the |\missingfigure| command. 
% The default value is |\textwidth|.
%
%
%
% \subsection{Options for the missingfigure command}
%
% \DescribeMacro{figwidth}
% The |figwidth=length| option sets the width of the figure inserted by the
% |\missingfigure| command.
% Length values below $6cm$ might trigger some problems with the
% visual appearance.
% Try to compare the default of the missing figure command, when the
% option is given or not. 
% \begin{verbatim}
% \missingfigure[figwidth=6cm]{Testing a long text string}
% \end{verbatim}
% \missingfigure[figwidth=6cm]{Testing a long text string}
% \begin{verbatim}
% \missingfigure{Testing a long text string}
% \end{verbatim}
% \missingfigure{Testing a long text string}
% \begin{wrapfigure}{r}[2cm]{6cm}
% \missingfigure[figwidth=6cm]{Add a test image \ldots}
% \end{wrapfigure}
% Another usage of the option is when |\missingfigure| is used in
% the wrapfigure environment.
% \begin{verbatim}
% \begin{wrapfigure}{r}[2cm]{6cm}
% \missingfigure[figwidth=6cm]{Add a test image \ldots}
% \end{wrapfigure}
% \end{verbatim}
% 
%
%
% \subsection{Options for the listoftodos command}
% The |\listoftodos| command takes one optional argument, that
% defines the name of the inserted list of todos.
% \begin{verbatim}
% \listoftodos[I can be called anything]
% \end{verbatim}
% 
%
%
% \subsection{Known issues}
% \subsubsection{Package loading order}
% The todonotes package requires the following packages.
% \begin{multicols}{2}
% \begin{itemize}
% \item ifthen
% \item xkeyval
% \item hyperref
% \item xcolor
% \item tikz
% \item calc
% \item graphicx (is loaded via the tikz package)
% \end{itemize}
% \end{multicols}
% \noindent
% When todonotes are loaded in the preamble, the package checks
% if these packages all are loaded. If that is not the case it loads
% the missing packages with no options given.
% If you want to give some specific options to some of these
% packages, you have to load them \emph{before} the todonotes
% package, otherwise you will get an ''Option clash'' error when
% latex works on the document.
%
%
% \subsubsection{Wrapping of long lines in list of todos}
% When a document is compiled with latex (and not pdflatex) long
% items in the list of todos are not wrapped into several lines, and
% do instead continue to the right out of the page.
%
%
% \subsubsection{Conflicts with the amsart documentclass}
% The |amsart| document class redefines some internal commands that
% is used by the todonotes package, this will cause an unfunctioning
% |\listoftodos| command.
% The following code to circumvent the problem was given by Dan
% Luecking on comp.text.tex
% \begin{verbatim}
% \makeatletter
%   \providecommand\@dotsep{5}
% \makeatother
% \listoftodos\relax 
% \end{verbatim}
% 
%
% \subsection{Things to improve}
% This is a list of things I consider to improve sometime in the
% future. I havent been done yet as I lack the time or skills to
% implement them.
% Patches with implementations of these tasks will be appreciated
% and might be included in the package if it will improve the
% package quality.
%
%
% \subsubsection{Owner information}
% Option for the todo command.
% \begin{verbatim}
% \todo[owner={Fabrice}]{Stuff}
% \end{verbatim}
% Add info on who ''owns'' the current todo. 
% Idea: Fabrice Niessen
%
%
% \subsubsection{Due date}
% Option for the todo command.
% \begin{verbatim}
% \todo[due=2008-12-07]{Stuff}
% \end{verbatim}
% Add info on when the current todo is due.
% Might be enhanced by a time line of the todos that have a due date
% assigned.
% Idea: Fabrice Niessen
%
%
% \subsubsection{Mark accomplished todos}
% \begin{verbatim}
% \todo[done]{Stuff}
% \end{verbatim}
% Idea: Fabrice Niessen
%
%
% \subsubsection{Enumeration of the todo notes}
% Add counters to the inserted notes.
% Idea: Henrik
%
%
%
% 
% 
% \subsection{Usage methods}
% In this section I have collected some different methods to use the
% |todonotes| package.
%
%
% \subsubsection{Define new commands with fixed options}
% Often there is a need for marking different classes of things to
% do (add reference, rewrite, \ldots).
% One way to do this, is to define some new commands as shown below
% (idea from Florent B.).
% \begin{verbatim}
% \newcommand{\addref}{\todo[color=red!40]{Add reference.}}
% \newcommand{\rewrite}[1]{\todo[color=green!40]{#1}}
% \end{verbatim}
% To distinguish between the different types of todos, the
% |todonotes| package can be loaded with the |colorinlistoftodos|
% option, which adds small colored squares to the list of todos.
% \begin{verbatim}
% \usepackage[colorinlistoftodos]{todonotes}
% \end{verbatim}
% 
% 
%
% \subsubsection{Combination with the fixme package}
%
% Thomas Arildsen has mailed me this use case. Check the
% documentation for the |fixme| package, as the code below relies
% directly on it (the |\FDUser| command is augumented when
% |\begin{document}| is reached).
% \begin{verbatim}
% \usepackage[user,nomargin]{fixme}
% \usepackage{todonotes}
% \newcommand{\FXUser}[2]{\todo[inline,size=\small]{{\bfseries #1:} #2}}
% \end{verbatim}
%
%
%
% \subsubsection{Altering the linespacing of todonotes}
%
% The |setspace| package lets you alter the line spacing of smaller
% sections of your document. The primary construct is the
% |spacing| environment, which is demonstrated below.
% \begin{verbatim}
% \begin{spacing}{0.5}
% Some lines with a decreased line spacing. This is accomplished
% using the setspace package that is included in standard latex
% distributions.
% \end{spacing}
% \end{verbatim}
% \begin{spacing}{0.5}
% Some lines with a decreased line spacing. This is accomplished
% using the setspace package that is included in standard latex
% distributions.
% \end{spacing}
%
% \vspace{0.25cm}
% Using the |spacing| environment we can define a new todonote
% command using the code below:
% \begin{verbatim}
% \newcommand{\smalltodo}[2][]
%     {\todo[caption={#2}, #1]
%     {\begin{spacing}{0.5}#2\end{spacing}}}
% \end{verbatim}
% \newcommand{\smalltodo}[2][]
%     {\todo[caption={#2}, #1]
%     {\begin{spacing}{0.5}#2\end{spacing}}}
% Todonotes with decreased line spacing can now be inserted as
% follows
% \smalltodo[size=\footnotesize]{
% Some lines with a decreased line spacing. This is accomplished
% using the setspace package that is included in standard latex
% distributions.}
% \begin{verbatim}
% \smalltodo[size=\footnotesize]{
% Some lines with a decreased line spacing. This is accomplished
% using the setspace package that is included in standard latex
% distributions.}
% \end{verbatim}
%
%
%
% \subsubsection{Comments ''a la Word''}
%
% Fabrice Niessen sent me the following usecase.
% The idea is to define a new command |\mycomment| which adds a
% counter and optionally the initials of the auther to the inserted
% todonote.
% 
% \begin{verbatim}
% \newcounter{mycomment}
% \newcommand{\mycomment}[2][]{%
%    % initials of the author (optional) + note in the margin
%    \refstepcounter{mycomment}%
%    {%
%        \setstretch{0.7}% spacing
%        \todo[color={red!100!green!33},size=\small]{%
%            \textbf{Comment [\uppercase{#1}\themycomment]:}~#2}%
%    }}
% \end{verbatim}
% \newcounter{mycomment}
% \newcommand{\mycomment}[2][]{%
%    \refstepcounter{mycomment}%
%    {%
%        \setstretch{0.7}%
%        \todo[color={red!100!green!33},size=\small]{%
%            \textbf{Comment [\uppercase{#1}\themycomment]:}~#2}%
%    }}
% 
% The command |\mycomment[HSM]{Testing first time.}| is displayed
% \mycomment[HSM]{Testing first time.}%
% like shown in the left margin, and another call of the command is
% added below |\mycomment[HSM]{Testing second time.}|.
% \mycomment[HSM]{Testing second time.}%
%
%
% \subsubsection{Marking new / old sections}
% Sometimes a whole section has to be marked by some means. You
% might want to try the following.
% \begin{verbatim}
% \todo[inline, caption={Some text}]{
% \begin{minipage}{\linewidth}
% Some text that might differ from the text given to the caption
% option.
% \end{minipage}
% }
% \end{verbatim}
% It is important to add the |caption={text}| option, otherwise
% latex will try to embed a minipage in the table of contents which
% triggers an error.
% Inside the minipage environment almost anything could be placed,
% except for other todo commands.
% \begin{verbatim}
% \todo[inline, caption={Examine this new section}]{
% \begin{minipage}{\linewidth}
% Some text.
% \begin{align}
% \sin(\theta)^2 + \cos(\theta)^2 = 1
% \end{align}
% A formula and a list
% \begin{itemize}
% \item	An item
% \end{itemize}
% \end{minipage}
% }
% \end{verbatim}
% The example above renders like
% \todo[inline, caption={Examine this new section}]{
% \begin{minipage}{\linewidth}
% Some text.
% \begin{align}
% \sin(\theta)^2 + \cos(\theta)^2 = 1
% \end{align}
% A formula and a list
% \begin{itemize}
% \item	An item
% \end{itemize}
% \end{minipage}
% }
%
%
% \iffalse
% \StopEventually{\PrintChanges\PrintIndex}
% \fi
% \StopEventually{\PrintChanges}
%
% \newpage
% \section{Implementation}
%
% Identifies the package and loads the packages dependences.
%    \begin{macrocode}
\ProvidesPackage{todonotes}[2009/01/09]
\RequirePackage{ifthen}
\RequirePackage{xkeyval}
\RequirePackage{hyperref}
\RequirePackage{xcolor}
\RequirePackage{tikz}
\RequirePackage{calc}
%    \end{macrocode}
% Some default values are set
%    \begin{macrocode}
\newcommand{\@todonotes@text}{}%
\newcommand{\@todonotes@backgroundcolor}{orange}
\newcommand{\@todonotes@linecolor}{orange}
\newcommand{\@todonotes@bordercolor}{black}
\newcommand{\@todonotes@textwidth}{\marginparwidth}
\newcommand{\@todonotes@textsize}{\normalsize}
\newcommand{\@todonotes@figwidth}{\textwidth}
%    \end{macrocode}
%
% \subsection{Declaration of options for the package}
% In this part the various options for
% the package are defined.

% Define the default text strings and set localization options for
% the danish and german languages.
%    \begin{macrocode}
\newcommand{\@todonotes@todolistname}{Todo list}
\newcommand{\@todonotes@MissingFigureText}{Figure}
\newcommand{\@todonotes@MissingFigureUp}{Missing}
\newcommand{\@todonotes@MissingFigureDown}{figure}
\newcommand{\@todonotes@SetTodoListName}[1]
    {\renewcommand{\@todonotes@todolistname}{#1}}
\newcommand{\@todonotes@SetMissingFigureText}[1]
    {\renewcommand{\@todonotes@MissingFigureText}{#1}}
\newcommand{\@todonotes@SetMissingFigureUp}[1]
    {\renewcommand{\@todonotes@MissingFigureUp}{#1}}
\newcommand{\@todonotes@SetMissingFigureDown}[1]
    {\renewcommand{\@todonotes@MissingFigureDown}{#1}}
\newif{\if@todonotes@reverseMissingFigureTriangle}
\DeclareOptionX{danish}{%
    \@todonotes@SetTodoListName{G\o{}rem\aa{}lsliste}%
    \@todonotes@SetMissingFigureText{Figur}%
    \@todonotes@SetMissingFigureUp{Manglende}%
    \@todonotes@SetMissingFigureDown{figur}%
}
\DeclareOptionX{german}{%
    \@todonotes@SetTodoListName{Liste der noch zu erledigenden Punkte}%
    \@todonotes@SetMissingFigureText{Abbildung}%
    \@todonotes@SetMissingFigureUp{Fehlende}%
    \@todonotes@SetMissingFigureDown{Abbildung}%
}
\DeclareOptionX{french}{%
    \@todonotes@SetTodoListName{Liste des points \`a traiter}%
    \@todonotes@SetMissingFigureText{Figure}%
    \@todonotes@SetMissingFigureUp{Figure}%
    \@todonotes@SetMissingFigureDown{manquante}%
    \@todonotes@reverseMissingFigureTrianglefalse
}
\DeclareOptionX{catalan}{
    \@todonotes@SetTodoListName{Llista de feines pendents}%
    \@todonotes@SetMissingFigureText{Figura}%
    \@todonotes@SetMissingFigureUp{Figura}%
    \@todonotes@SetMissingFigureDown{pendent}%
}
\DeclareOptionX{spanish}{
    \@todonotes@SetTodoListName{Lista de tareas pendientes}%
    \@todonotes@SetMissingFigureText{Figura}%
    \@todonotes@SetMissingFigureUp{Figura}%
    \@todonotes@SetMissingFigureDown{pendient}%
}
%    \end{macrocode}
% Create a counter, for storing the number of inserted todos.
%    \begin{macrocode}
\newcounter{@todonotes@numberoftodonotes}
%    \end{macrocode}
%
% Make it possible to disable the functionality
% of the package. If this option is given, the
% commands  |\todo{}| and |\listoftodos| are defined
% as commands with no effect. (But you can still
% compile you document with these commands).
%    \begin{macrocode}
\newif{\if@todonotes@disabled}
\DeclareOptionX{disable}{\@todonotes@disabledtrue}
%    \end{macrocode}
% Show small boxes in the list of todos with the color of the
% inserted todonotes.
%    \begin{macrocode}
\newif{\if@todonotes@colorinlistoftodos}
\DeclareOptionX{colorinlistoftodos}{\@todonotes@colorinlistoftodostrue}
%    \end{macrocode}
% The default style behaves bad when compiled
% with latex (some text placement problems).
% The dvistyle option, changes the visual behavior
% to avoid this text placement problem.
%    \begin{macrocode}
\newif{\if@todonotes@dviStyle}
\DeclareOptionX{dvistyle}{\@todonotes@dviStyletrue}
%    \end{macrocode}
% Create a color option.
%    \begin{macrocode}
\define@key{todonotes.sty}%
    {color}{
        \renewcommand{\@todonotes@backgroundcolor}{#1}
        \renewcommand{\@todonotes@linecolor}{#1}}
%    \end{macrocode}
% Make the background color of the notes as
% an option.
%    \begin{macrocode}
\define@key{todonotes.sty}%
    {backgroundcolor}{\renewcommand{\@todonotes@backgroundcolor}{#1}}
%    \end{macrocode}
% Make the line color of the notes as
% an option.
%    \begin{macrocode}
\define@key{todonotes.sty}%
    {linecolor}{\renewcommand{\@todonotes@linecolor}{#1}}
%    \end{macrocode}
% Make the color of the notes box color as
% an option.
%    \begin{macrocode}
\define@key{todonotes.sty}%
    {bordercolor}{\renewcommand{\@todonotes@bordercolor}{#1}}
%    \end{macrocode}
% Set whether short captions given as arguments to the todo command
% should be included in the inserted todonote.
%    \begin{macrocode}
\newif{\if@todonotes@prependcaptionglobal}
\@todonotes@prependcaptionglobalfalse
\DeclareOptionX{prependcaption}{\@todonotes@prependcaptionglobaltrue}
%    \end{macrocode}
% Make the text width as an option.
%    \begin{macrocode}
\define@key{todonotes.sty}%
    {textwidth}{\renewcommand{\@todonotes@textwidth}{#1}}
%    \end{macrocode}
% Make the text size as an option. It requires some magic with the 
% |\csname| and |\endcsname| macros, as commands cannot be taken as
% options for a package.
%    \begin{macrocode}
\define@key{todonotes.sty}%
    {textsize}{\renewcommand{\@todonotes@textsize}{\csname #1\endcsname}}
%    \end{macrocode}
% Add option for shadows behind the inserted notes
%    \begin{macrocode}
\newif{\if@todonotes@shadowenabled}
\@todonotes@shadowenabledfalse
\DeclareOptionX{shadow}{\@todonotes@shadowenabledtrue
\usetikzlibrary{shadows}}
%    \end{macrocode}
% Add option for the default width of the figure inserted with
% |\missingfigure|.
%    \begin{macrocode}
\define@key{todonotes.sty}%
    {figwidth}{\renewcommand{\@todonotes@figwidth}{#1}}
%    \end{macrocode}
% Make the text width as an option.
%    \begin{macrocode}
% Finally process the given options.
%    \begin{macrocode}
\ProcessOptionsX
%    \end{macrocode}


% \subsection{Options for the todo command}
%
% In this part the various options for 
% commands in the package are defined.
% Set an arbitrarily fill color
%    \begin{macrocode}
\newcommand{\@todonotes@currentlinecolor}{}%
\newcommand{\@todonotes@currentbackgroundcolor}{}%
\newcommand{\@todonotes@currentbordercolor}{}%
\define@key{todonotes}{color}{%
    \renewcommand{\@todonotes@currentlinecolor}{#1}%
    \renewcommand{\@todonotes@currentbackgroundcolor}{#1}}%
\define@key{todonotes}{linecolor}{%
    \renewcommand{\@todonotes@currentlinecolor}{#1}}%
\define@key{todonotes}{backgroundcolor}{%
    \renewcommand{\@todonotes@currentbackgroundcolor}{#1}}%
\define@key{todonotes}{bordercolor}{%
    \renewcommand{\@todonotes@currentbordercolor}{#1}}%
%    \end{macrocode}
% Set a relative font size
%    \begin{macrocode}
\newcommand{\@todonotes@sizecommand}{}%
\define@key{todonotes}{size}{\renewcommand{\@todonotes@sizecommand}{#1}}%
%    \end{macrocode}
% Should the todo item be included in the list of todos?
%    \begin{macrocode}
\newif\if@todonotes@appendtolistoftodos%
\define@key{todonotes}{list}[]{\@todonotes@appendtolistoftodostrue}%
\define@key{todonotes}{nolist}[]{\@todonotes@appendtolistoftodosfalse}%
%    \end{macrocode}
% Should the todo item be displayed inline?
%    \begin{macrocode}
\newif\if@todonotes@inlinenote%
\define@key{todonotes}{inline}[]{\@todonotes@inlinenotetrue}%
\define@key{todonotes}{noinline}[]{\@todonotes@inlinenotefalse}%
%    \end{macrocode}
%    \begin{macrocode}
\newif\if@todonotes@prependcaption%
\define@key{todonotes}{prepend}[]{\@todonotes@prependcaptiontrue}%
\define@key{todonotes}{noprepend}[]{\@todonotes@prependcaptionfalse}%
%    \end{macrocode}
% Should the note in the margin be connected to the insertion point
% in the text?
%    \begin{macrocode}
\newif\if@todonotes@line%
\define@key{todonotes}{line}[]{\@todonotes@linetrue}%
\define@key{todonotes}{noline}[]{\@todonotes@linefalse}%
%    \end{macrocode}
% Should the text in the list of todos be different from the text
% in the todonote?
%    \begin{macrocode}
\newcommand{\@todonotes@caption}{}%
\newif\if@todonotes@captiongiven%
\define@key{todonotes}{caption}%
    {\renewcommand{\@todonotes@caption}{#1}%
    \@todonotes@captiongiventrue}%
\define@key{todonotes}{nocaption}[]{\@todonotes@captiongivenfalse}%
%    \end{macrocode}
% Change the current figure width.
%    \begin{macrocode}
\newcommand{\@todonotes@currentfigwidth}{\@todonotes@figwidth}
\define@key{todonotes}%
    {figwidth}{\renewcommand{\@todonotes@currentfigwidth}{#1}}
%    \end{macrocode}
% Preset values of the options
%    \begin{macrocode}
\presetkeys%
    {todonotes}%
    {linecolor=\@todonotes@linecolor,%
    backgroundcolor=\@todonotes@backgroundcolor,%
    bordercolor=\@todonotes@bordercolor,%
    noinline,%
    nocaption,%
    figwidth=\@todonotes@figwidth,%
    line, list, size=\@todonotes@textsize}{}%
%    \end{macrocode}
% \subsection{The main code part}
% Here is the actual macros defined.
% If the option "disable" was passed to the package
% define empty commands.
%    \begin{macrocode}
\if@todonotes@disabled%
    \newcommand{\listoftodos}[1][]{}
    \newcommand{\todo}[2][]{\ignorespaces}
    \newcommand{\missingfigure}[2]{}
\else % \if@todonotes@disabled
%    \end{macrocode}
% Define the |\listoftodos| command and define the 
% appearance of the list of todos.
%    \begin{macrocode}
\newcommand{\listoftodos}[1][\@todonotes@todolistname]
    {\section*{#1} \@starttoc{tdo}}
\newcommand{\l@todo}
    {\@dottedtocline{1}{0em}{2.3em}}
%    \end{macrocode}
% Define styles used by the todo command
%    \begin{macrocode}
\tikzstyle{notestyleraw} = [
    draw=\@todonotes@currentbordercolor, 
    fill=\@todonotes@currentbackgroundcolor,
    line width=0.5pt,
    text width = \@todonotes@textwidth - 1.6 ex - 1pt, 
    inner sep = 0.8 ex,
    rounded corners=4pt]
%    \end{macrocode}
% Add shadows and rounded corners to the inserted todonotes.
%    \begin{macrocode}
\if@todonotes@shadowenabled
\tikzstyle{notestyle} = [notestyleraw,
    general shadow={shadow xshift=.5ex, shadow yshift=-.5ex,
        opacity=1,fill=black!50}]
\else
\tikzstyle{notestyle} = [notestyleraw]
\fi
\tikzstyle{notestyleleft} = [
    notestyle, 
    left]
\tikzstyle{connectstyle} = [
    thick, 
    draw=\@todonotes@currentlinecolor]
\tikzstyle{inlinenotestyle} = [
    notestyle, 
    text width=\linewidth - 1.6 ex - 1 pt]
%    \end{macrocode}
%
% \begin{macro}{\todo}
% Define the todo command
%    \begin{macrocode}
\newcommand{\todo}[2][]{%
%    \end{macrocode}
% Use the global value for determining the default prepend behavior.
%    \begin{macrocode}
\if@todonotes@prependcaptionglobal%
	\@todonotes@prependcaptiontrue%
\else%
	\@todonotes@prependcaptionfalse%
\fi%
%    \end{macrocode}
% Store the original text for later usage.
%    \begin{macrocode}
\renewcommand{\@todonotes@text}{#2}%
\renewcommand{\@todonotes@caption}{#2}%
\setkeys{todonotes}{#1}%
%    \end{macrocode}
% Add the item to the list of todos. When the option
% |colorinlistoftodos| is given to the package a small colored 
% square is added in front of the text.
%    \begin{macrocode}
\addtocounter{@todonotes@numberoftodonotes}{1}%
\if@todonotes@appendtolistoftodos%
    \phantomsection%
    \if@todonotes@captiongiven%
    \else%
        \renewcommand{\@todonotes@caption}{#2}%
    \fi%
    \if@todonotes@colorinlistoftodos%
        \addcontentsline{tdo}{todo}{\protect{%
            \colorbox{\@todonotes@currentbackgroundcolor}%
                {\textcolor{\@todonotes@currentbackgroundcolor}{o}}%
            \ \@todonotes@caption}}%
    \else%
        \addcontentsline{tdo}{todo}{\protect{\@todonotes@caption}}%
    \fi%
\fi%
%    \end{macrocode}
% Prepend the short caption given if it is requested
%    \begin{macrocode}
\if@todonotes@captiongiven%
    \if@todonotes@prependcaption%
        \renewcommand{\@todonotes@text}{\@todonotes@caption: #2}%
    \fi%
\fi%
%    \end{macrocode}
% Place the todonote as indicated by the options (inline or in a
% marginpar), below is the code for the inline placement.
%    \begin{macrocode}
\if@todonotes@inlinenote%
    \if@todonotes@dviStyle%
        {\par\noindent\begin{tikzpicture}[remember picture]%
            \draw node[inlinenotestyle] {};\end{tikzpicture}\par}%
        {\noindent \@todonotes@sizecommand \@todonotes@text}%
        {\par\noindent\begin{tikzpicture}[remember picture]%
            \draw node[inlinenotestyle] {};\end{tikzpicture}\par}%
    \else%
        {\par\noindent\begin{tikzpicture}[remember picture]%
            \draw node[inlinenotestyle] {\@todonotes@sizecommand \@todonotes@text};%
            \end{tikzpicture}\par}%
    \fi%
\else%
%    \end{macrocode}
% When the todonote should be placed inside a marginpar, the code
% below is applied.
% First is the current location in the document stored, this enables
% us later to connect this point with the inserted todonote. 
%    \begin{macrocode}
\begin{tikzpicture}[remember picture, baseline=-0.75ex]%
    \node [coordinate] (inText) {};%
\end{tikzpicture}%
\marginpar[{% Draw note in left margin
\if@todonotes@dviStyle%
    \begin{tikzpicture}[remember picture]%
        \draw node[notestyle] {};%
    \end{tikzpicture}\\ %
    \begin{minipage}{\@todonotes@textwidth}%
    \@todonotes@sizecommand \@todonotes@text%
    \end{minipage}\\%
    \begin{tikzpicture}[remember picture]%
        \draw node[notestyle] (inNote) {};%
    \end{tikzpicture}%
\else%
    \begin{tikzpicture}[remember picture]%
        \draw node[notestyle] (inNote)%
            {\@todonotes@sizecommand \@todonotes@text};%
    \end{tikzpicture}%
\fi%
\if@todonotes@line%
    \begin{tikzpicture}[remember picture, overlay]%
        \draw[connectstyle]%
        ([yshift=-0.2cm] inText)%
        -| ([xshift=0.2cm] inNote.east)%
        -| (inNote.east);%
    \end{tikzpicture}%
\fi%
%    \end{macrocode}
% In the book documentclass (which is a twoside layout), the
% |\marginpar| marginpar macro takes two arguments
% |\marginpar[left]{right}|.
% If both arguments are given, latex will decide in which side the
% margin note has to be inserted, and then use the corresponding
% commands.
%    \begin{macrocode}
}]{% Draw note in right margin
\if@todonotes@dviStyle%
    \begin{tikzpicture}[remember picture]%
        \draw node[notestyle] {};%
    \end{tikzpicture}\\%
    \begin{minipage}{\@todonotes@textwidth}%
    \@todonotes@sizecommand \@todonotes@text%
    \end{minipage}\\%
    \begin{tikzpicture}[remember picture]%
        \draw node[notestyle] (inNote) {};%
    \end{tikzpicture}%
\else%
    \begin{tikzpicture}[remember picture]%
        \draw node[notestyle](inNote)%
            {\@todonotes@sizecommand \@todonotes@text};%
    \end{tikzpicture}%
\fi%
\if@todonotes@line%
    \begin{tikzpicture}[remember picture, overlay]%
        \draw[connectstyle]%
        ([yshift=-0.2cm] inText)%
        -| ([xshift=-0.2cm] inNote.west)%
        -| (inNote.west);%
    \end{tikzpicture}%
\fi%
}%
\fi%
\ignorespaces%
}%
%    \end{macrocode}
% \end{macro}
% \begin{macro}{\missingfigure}
% Defines the |\missingfigure| macro.
%    \begin{macrocode}
\newcommand{\missingfigure}[2][]{
\setkeys{todonotes}{#1}%
\addcontentsline{tdo}{todo}{\@todonotes@MissingFigureText: \protect{#2}}%
\par
\noindent
\begin{tikzpicture}
\draw[fill=black!40, draw = white, line width=0pt] 
    (-2, -2.5) rectangle +(\@todonotes@currentfigwidth, 4cm);
\draw (2, -0.3) node[right, text
    width=\@todonotes@currentfigwidth-4.5cm] {#2};
\draw[red, fill=white, rounded corners = 5pt, line width=10pt] 
    (30:2cm) -- (150:2cm) -- (270:2cm) -- cycle;
\draw (0, 0.3) node {\@todonotes@MissingFigureUp};
\draw (0, -0.3) node {\@todonotes@MissingFigureDown};
\end{tikzpicture}
}% Ending \missingfigure command
\fi % Ending \@todonotes@ifdisabled
%    \end{macrocode}
% \end{macro}
%
% \Finale
\endinput

